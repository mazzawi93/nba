\chapter{Basketball Model}

\section{Introduction}
This section will explain the formulation of the model for the NBA.  The model begins by following the Dixon and Coles model and expanding into the Dixon and Robinson model.  It is based off of the Dixon and Robinson model for English football.  The first thing to do was implement the Dixon and Coles model as this is the base of the Dixon and Robinson model.  \ldots  This is not needed as basketball is a very scoring game and these variables will never be used.



The numbers can be seen in Table \ref{table:dc_141516}.  Using these numbers, and selecting the team with the highest probability of winning a game.  The accuracy of selecting a winner was 63.84\% and the value betting strategy won 57.84 units with 37.06\% winning bets.

Tried the Dixon and Coles model with a time parameter of 0.02 and the values seem more accurate.  The accuracy of selecting a winner was 63.21\% and the value betting strategy won 63.55 units with 38.03\% betting accuracy.  Did only the 2016 season and 64\% with the new vectorized way of doing it and 63.25 units.

Vectorized Dixon Coles and went from like 1341 seconds to 9 seconds for 3 seasons worth.  For some reason when I do Dixon Coles with the time factor with 2014, 2015 it gives me weird numbers but works perfectly fine with 2016 and 2017.  Must be something with the dataset

\section{Some Math Stuff}

In the Dixon-Robinson model,.  By looking at NBA statistics the final minute of each quarter is likely to have an increase in scoring.


Dixon and Robinson model 1 is shown in Table \ref{table:dr_1_141516}.  The accuracy of selecting a winner was 60.28\% and the value betting strategy won 98.14 units with 37.31\% betting accuracy.

The issue with the betting is that it likes to bet on very high odds as the model computes probabilities differently.  Odds over 5 are usually bet.


\begin{equation}
    \lambda_k(t) = 
    \begin{cases}
        \rho_{1}\lambda_{xy}\lambda_{k} & \text{for }t \in (11/48, 12/48],\\
        \rho_{2}\lambda_{xy}\lambda_{k} & \text{for }t \in (23/48, 24/48],\\
        \rho_{3}\lambda_{xy}\lambda_{k} & \text{for }t \in (35/48, 36/48],\\
        \rho_{4}\lambda_{xy}\lambda_{k} & \text{for }t \in (47/48, 48/48],\\
        \lambda_{xy}\lambda_{k} & \text{otherwise}
\end{cases}
\end{equation}

The away rate is similar $\mu_k(t)$.

\begin{table}[h]
\centering
\caption{My caption}
\label{my-label}
\begin{tabular}{|l|l|}
\hline
0     & -813.8585 \\ \hline
0.005 & -808.1008 \\ \hline
0.01 & -804.0550\\ \hline-813.8585
0.015 & -801.5937\\ \hline
0.02 & -800.4334\\ \hline
0.025 & -800.2177\\ \hline
0.03 & -800.5793 \\ \hline
0.035 & -801.3159 \\ \hline
0.04 & -802.1665 \\ \hline
0.045 & \\ \hline
0.05 & \\ \hline

\end{tabular}
\end{table}

The optimal value is 0.024.

\section{Team Model}

Just like the model from Section a, with $n$ teams, the offence parameters $\{\alpha_1,\ldots,\alpha_n\}$ $\{\beta_1,\ldots,\beta_n\}$ $\gamma_h$ are to be estimated.  The following constraint is imposed due to the 

$$\frac{1}{n}\sum^{n}_{i=1}\alpha_i = 100$$

For the NBA, there are 30 teams, which gives a total of 61 parameters to be estimated.

\section{Player Model}