\chapter{Background}

In this chapter, all of the background knowledge required for the project will be discussed.

This chapter will be a review for\cite{theoretical_stat}

\section{Basketball}
Basketball is a popular sport played by teams with 5 players on each side.  Seasons are played across two years, beginning in October and ending in April of the following year.  For the purposes of this paper, when a season is discussed for example the 2016 season, it will be in reference to the 2015-16 season.

There are 5 players on court per team, each with their own responsibilities summarized below by position \cite{player_criteria}:

\begin{itemize}
	\item \textbf{Point Guard:} They are the play maker, the ball is usually in their hands.  They organize the team's play and control the intensity of play.
	\item Shooting Guard:
	\item Small Forward
	\item Power Forward
	\item Centre 
\end{itemize}
Position 1 – level of defensive pressure, transition defense efficiency, the ball control,
passing skills, dribble penetration, outside shots, and transition offence efficiency;
Position 2 – level of defensive pressure, transition defense efficiency, outside shots,
dribble penetration, offence without the ball, and transition offence efficiency;
Position 3 – transition defense efficiency, outside shots, dribble penetration, offense
without the ball, free throws, and transition offence efficiency;
Position 4 – defensive and offensive rebounding efficiency, inside shots, dribble penetration,
efficiency of screening, and free throws;
Position 5 – defensive and offensive rebounding efficiency, inside shots, dribble penetration,
efficiency of screening, drawing fouls and three-point plays, and free throws
When selecting a player for the fantasy football team, the positions will come under different scrutiny 

\section{Fantasy Basketball}


\subsection{Scoring Types}
There are two main scoring types for fantasy basketball leagues: Rotisserie and Head-to-head.

\section{Mathematical Concepts}

\subsection{Poisson Distribution}
The Poisson distribution is defined by the formula \cite{poisson}

\begin{equation}
p_x(\lambda) = \frac{e^{-\lambda}\lambda^x}{x!}
\end{equation}